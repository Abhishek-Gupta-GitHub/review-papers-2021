 Consideration of pebbles(cm and m sized) for planet formation has gained importance as it solves two problems: (i) Formation of planetesimals; and (ii) long formation time scales of protoplanetary cores to start gas accretion. Problem (i) is related to the difficulty in forming km-sized objects due to a barrier called metre size barrier. The planetisimals overcome this barrier by either streaming instability or gravitational instability. The influx of pebbles at this stage help in growth of protoplanetary cores. The cross section of the emrbyos for pebble accretion is few orders larger than that for km-sized objects. This results in faster planetary growth with pebble accretion compared to classical model which solves problem (ii).\cite{refId0} employs a symplectic N-body integrator SyMBA to perform numerical simulations to study planet formation in protostellar discs via pebble accretion by changing parameters such as the stellar metallicity, the disc mass, and the disc’s viscosity.
