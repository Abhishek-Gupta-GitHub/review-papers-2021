\begin{enumerate}

\item It was observed that only the massive, low-viscosity and metal-rich discs favoured the fastest giant planet formation while Es and SEs are formed more easily. The higher viscosity leads to the faster disc evolution and the lower pebble mass flux, and thus these lower planet formation. In both migration and no-migration simulations, 60\% of the planets had eccentricities $<$ 0.1 and inclinations $\sim$ $0^o$. The simulations showed that these parameters are lower for higher planet multiplicity systems. Though these trends were consistent with the observations, they were different from actual distributions. Another observation was that the formation of giant planets in the simulations was almost 50\% within $\sim$ 1.5 Myr and $\sim$ 90\% within $\sim$ 3 Myr (Figure 1). Thus, giant planet formation with pebble accretion was faster than in the classical formation model and consistent with time scales of Jupiter’s formation.

\begin{figure}[H] 
\centering
\includegraphics[height=8cm,width=11cm]{pebble accretion/fig 4.jpg}

\caption{Cumulative plot of formation timescales of a giant planet with mass $\geq$0.1 $M_J$ with (orange) and without migration cases (blue).\cite{refId0}}

\end{figure}

\item It was seen that greater the metallicity of the stars, faster is the rate of formation of Es/SEs and a greater number of them is formed. However, the dependency of metallicity for Es$/$SEs is subtle, because the final outcomes depend on the timing of planet formation as well as the later dynamical evolution of planets.

\item In the simulations without migration,planets with $M_E$ had $\leq$ 1\% mass in their envelope at the time of formation. The mass of SEs in their core and envelope was approximately the same.Giant planets had few to several masses in their core.However, by the end of 50 Myr, these figures changed drastically due to planet-planet interactions and dynamic evolution. Similar results were obtained in simulations with migration but the variation in densities appeared faster.

\item It was observed that dynamical instability events occurred when the gas disc was $\leq$ 4 Myr in cases without migration, and when the gas disc dissipated in cases with migration. The occurrence of dynamic instabilities during the migration phase may be the reason for the orbital eccentricities and inclinations to be lower in the former case. In cases of no-migration, the production rate of free-floating planets and merger rate was low for giant planets($\sim$ 1.94\% and $\sim$ 1.98\% of the ejected planets) and significant for Es and SEs($\sim$ 48.9\% and $\sim$ 31.3\%). However, in cases of migration,$\sim$ 96.7\% of merged planets were Es/SEs  while $\sim$ 1.27\% of merged planets were giant planets. No obvious dependence between the total masses of removed planets with the survived planets was found.Based on the simulations, the presence of one E/SE free-foating planet for every two Mars-like planets would be expected.

\item The planetary system in the simulations were compact and similar to Kepler's multi-planetary system. This is in accordance to the low eccentricities of planets found in these simulations. A thing to be noted is that these results include only those systems in which planetary formation proceeds sufficiently so that all planets have masses $\geq$0.1 ME.


\end{enumerate}

