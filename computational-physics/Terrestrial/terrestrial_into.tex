Focusing  on the  last stage the formation process starting from the end product of the intermediate stage is studied. It necessitated a good model for the final configuration of the second stage of planetary formation, which is then taken as initial conditions.Two different treatments i) Analytical models and ii) Numerical models has been followed both of which investigated the dynamical evolution of the planetesimal population using statistical methods.The main idea behind these methods is to study the evolution of the size and velocity distribution of the protoplanets ( two aspects which are closely related).\\


There seems to be no general agreement on the final stages of the planetesimal population at the second stage ending as of yet, Whether or not runaway accretion occurred during this period being the principal question. Work by so-called Moscow and Kyoto'schools' on this stage of accumulation found insignificant tendency for one or more planetesimals to become detached from a  continuous power-law mass distribution and collect most of the material available. The expected final state of the intermediate stage from these results would be a rather homogeneous mass distribution.Stewart \& Wetherill (1988), in contrast, found runaway accretion in their models, leading to a configuration with a few embryos, while most of the mass remained in the size range of the original bodies.
The opposing results is to do with the way in which each approach treats gravitational perturbations, and the emphasis on different physical processes present during this period (gas drag, collisions etc.).The question of whether we should expect a fairly homogeneous mass distribution or  runaway type one is still unresolved even though recent studies point to significant mass spectrum. The choice of the initial conditions might be crucial to final outcome, the first alternative is adopted as the simplest working hypothesis in order to illustrate our method setting the scene of the final stage of terrestrial planetary formation.\\

First serious attempt of the direct N-body simulation of the final step in the formation of the terrestrial planets was by Cox and Lewis(1980). Starting with swarm of 100 bodies of equal mass, $m = M_{\oplus}/50$, distributed uniformly from 0.5 to 1.5  AU in 2D, with $e_{max}$ range varying between different models from 0.025 to 0.15 and initial eccentricities chosen randomly between $0$ to $e_{max}$.Even in the model with  $e_{max} = 0.1$ the isolation of the feeding zones of the embryos was found to be occuring with the final configuration containing too many bodies/planets as a consequence henceforth not representing the real system. Only at high initial maximum eccentricities ($\approx 0.15$) did the number of bodies ended up lower.Due to  gas drag effects of the primitive solar nebula present during the second stage, as well as the circularisation of the orbit accompanying the accretion process it is doubtful that such high eccentricities at the initial stages is possible.

The Cox and Lewis (1980) model simplifies the gravitational perturbations of the planetesimals by assuming that the planetesimals move in unperturbed Keplerian orbit until they pass within the sphere of influence, $R_{S}$, of another body, after which perturbations were included. Therefore, whether more complete treatment of the dynamics would improve the results is the question.
Lecar \& Aarseth (1986), laying Stress on this point performed a 2-D simulation of 200 lunar sized bodies distributed uniformly between 0.5 \& 1.5 AU and initial  
eccentricities set to 0. The mutual gravitational perturbations were treated more carefully by dividing the orbital plane into 100 azimuthal zones (bins) centred on the Sun, and employing eight
radial zones. The perturbations considered
on each planetesimal were due to bodies selected by a neighbour scheme which, initially included the effect of distant encounters out to $300 R_{s}$, and this limit was gradually extended as the total number of bodies became smaller and all the perturbations were included when the number reached 10.
The importance of including long range perturbations were shown in the results of Lecar \& Aarseth (1986).
Even with initial circular orbits, they found no occurrence of premature isolation of bodies.An
apparently stable configuration of six bodies remained after approximately $6\times 10^{4} yr$,the largest having about two-thirds of the Earth's mass.However, this model is far from representing the real state of
the terrestrial planet as the final number of bodies was somewhat too high and there was no good agreement of the orbital characteristics with the terrestrial planets (eccentricities were too high, and the distribution of the semi-major axis did not agree well with that of the actual planetary system). 
A N-body simulation of this stage of formation of the terrestrial planets (Mercury, Venus, Earth and Mars) is performed in this paper\cite{1990MNRAS.245...30B} based on the Lecar \& Aarseth (1986) model with a more detailed and realistic model for collision outcomes that allows fragmentation and cratering along with the accretion of the bodies.
